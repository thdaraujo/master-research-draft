% Arquivo LaTeX de exemplo de dissertação/tese a ser apresentados à CPG do IME-USP
%
% Versão 5: Sex Mar  9 18:05:40 BRT 2012
%
% Criação: Jesús P. Mena-Chalco
% Revisão: Fabio Kon e Paulo Feofiloff
%
% Obs: Leia previamente o texto do arquivo README.txt

\documentclass[11pt,twoside,a4paper]{book}

% ---------------------------------------------------------------------------- %
% Pacotes
\usepackage[T1]{fontenc}
\usepackage[brazil]{babel}
\usepackage[utf8]{inputenc}
\usepackage[pdftex]{graphicx}           % usamos arquivos pdf/png como figuras
\usepackage{setspace}                   % espaçamento flexível
\usepackage{indentfirst}                % indentação do primeiro parágrafo
\usepackage{makeidx}                    % índice remissivo
\usepackage[nottoc]{tocbibind}          % acrescentamos a bibliografia/indice/conteudo no Table of Contents
\usepackage{courier}                    % usa o Adobe Courier no lugar de Computer Modern Typewriter
\usepackage{type1cm}                    % fontes realmente escaláveis
\usepackage{listings}                   % para formatar código-fonte (ex. em Java)
\usepackage{titletoc}
%\usepackage[bf,small,compact]{titlesec} % cabeçalhos dos títulos: menores e compactos
\usepackage[fixlanguage]{babelbib}
\usepackage[font=small,format=plain,labelfont=bf,up,textfont=it,up]{caption}
\usepackage[usenames,svgnames,dvipsnames]{xcolor}
\usepackage[a4paper,top=2.54cm,bottom=2.0cm,left=2.0cm,right=2.54cm]{geometry} % margens
%\usepackage[pdftex,plainpages=false,pdfpagelabels,pagebackref,colorlinks=true,citecolor=black,linkcolor=black,urlcolor=black,filecolor=black,bookmarksopen=true]{hyperref} % links em preto
\usepackage[pdftex,plainpages=false,pdfpagelabels,pagebackref,colorlinks=true,citecolor=DarkGreen,linkcolor=NavyBlue,urlcolor=DarkRed,filecolor=green,bookmarksopen=true]{hyperref} % links coloridos
\usepackage[all]{hypcap}                    % soluciona o problema com o hyperref e capitulos
\usepackage[round,sort,nonamebreak]{natbib} % citação bibliográfica textual(plainnat-ime.bst)
\fontsize{60}{62}\usefont{OT1}{cmr}{m}{n}{\selectfont}

% ---------------------------------------------------------------------------- %
% Cabeçalhos similares ao TAOCP de Donald E. Knuth
\usepackage{fancyhdr}
\pagestyle{fancy}
\fancyhf{}
\renewcommand{\chaptermark}[1]{\markboth{\MakeUppercase{#1}}{}}
\renewcommand{\sectionmark}[1]{\markright{\MakeUppercase{#1}}{}}
\renewcommand{\headrulewidth}{0pt}

% ---------------------------------------------------------------------------- %
\graphicspath{{./figuras/}}             % caminho das figuras (recomendável)
\frenchspacing                          % arruma o espaço: id est (i.e.) e exempli gratia (e.g.)
\urlstyle{same}                         % URL com o mesmo estilo do texto e não mono-spaced
\makeindex                              % para o índice remissivo
\raggedbottom{}                           % para não permitir espaços extra no texto
\fontsize{60}{62}\usefont{OT1}{cmr}{m}{n}{\selectfont}
\cleardoublepage{}
\normalsize

% ---------------------------------------------------------------------------- %
% Opções de listing usados para o código fonte
% Ref: http://en.wikibooks.org/wiki/LaTeX/Packages/Listings
\lstset{ %
language=Java,                  % choose the language of the code
basicstyle=\footnotesize,       % the size of the fonts that are used for the code
numbers=left,                   % where to put the line-numbers
numberstyle=\footnotesize,      % the size of the fonts that are used for the line-numbers
stepnumber=1,                   % the step between two line-numbers. If it's 1 each line will be numbered
numbersep=5pt,                  % how far the line-numbers are from the code
showspaces=false,               % show spaces adding particular underscores
showstringspaces=false,         % underline spaces within strings
showtabs=false,                 % show tabs within strings adding particular underscores
frame=single,	                % adds a frame around the code
framerule=0.6pt,
tabsize=2,	                    % sets default tabsize to 2 spaces
captionpos=b,                   % sets the caption-position to bottom
breaklines=true,                % sets automatic line breaking
breakatwhitespace=false,        % sets if automatic breaks should only happen at whitespace
escapeinside={\%*}{*)},         % if you want to add a comment within your code
backgroundcolor=\color[rgb]{1.0,1.0,1.0}, % choose the background color.
rulecolor=\color[rgb]{0.8,0.8,0.8},
extendedchars=true,
xleftmargin=10pt,
xrightmargin=10pt,
framexleftmargin=10pt,
framexrightmargin=10pt
}

% ---------------------------------------------------------------------------- %
% Corpo do texto
\begin{document}
\frontmatter
% cabeçalho para as páginas das seções anteriores ao capítulo 1 (frontmatter)
\fancyhead[RO]{{\footnotesize\rightmark}\hspace{2em}\thepage}
\setcounter{tocdepth}{2}
\fancyhead[LE]{\thepage\hspace{2em}\footnotesize{\leftmark}}
\fancyhead[RE,LO]{}
\fancyhead[RO]{{\footnotesize\rightmark}\hspace{2em}\thepage}

\onehalfspacing{}  % espaçamento

% ---------------------------------------------------------------------------- %
% Variáveis

\newcommand{\mydocumenttitle}{Uma Abordagem Baseada em Ontologias para a Predição de Ligações em Redes de Colaboração Científica}


% ---------------------------------------------------------------------------- %
% CAPA
% Nota: O título para as dissertações/teses do IME-USP devem caber em um
% orifício de 10,7cm de largura x 6,0cm de altura que há na capa fornecida pela SPG.
\thispagestyle{empty}
\begin{center}
    \vspace*{2.3cm}
    \textbf{\Large{\mydocumenttitle{}}}\\

    \vspace*{1.2cm}
    \Large{Thiago Henrique Dias Araujo}

    \vskip 2cm
    \textsc{
    Texto apresentado\\[-0.25cm]
    ao\\[-0.25cm]
    Instituto de Matemática e Estatística\\[-0.25cm]
    da\\[-0.25cm]
    Universidade de São Paulo\\[-0.25cm]
    para\\[-0.25cm]
    o Exame de Qualificação\\[-0.25cm]
    de\\[-0.25cm]
    Mestre em Ciências}

    \vskip 1.5cm
    Programa: Mestrado em Ciência da Computação\\
    Orientadora: Profª. Drª. Renata Wassermann

   	\vskip 1cm

    \vskip 0.5cm
    \normalsize{São Paulo, março de 2016}
\end{center}

% ---------------------------------------------------------------------------- %
% Página de rosto (SÓ PARA A VERSÃO DEPOSITADA - ANTES DA DEFESA)
% Resolução CoPGr 5890 (20/12/2010)
%
% IMPORTANTE:
%   Coloque um '%' em todas as linhas
%   desta página antes de compilar a versão
%   final, corrigida, do trabalho
%
%
\newpage
\thispagestyle{empty}
    \begin{center}
        \vspace*{2.3 cm}
        \textbf{\Large{\mydocumenttitle{}}}\\
        \vspace*{2 cm}
    \end{center}

    \vskip 2cm

    \begin{flushright}
	Esta é a versão original do texto elaborado pelo\\
	candidato Thiago Henrique Dias Araujo para o\\
  exame de qualificação apresentado ao Instituto\\
  de Matemática e Estatística da Universidade de\\
  São Paulo como requisito para obtenção de título\\
  de Mestre em Ciências.
    \end{flushright}

\pagebreak


% ---------------------------------------------------------------------------- %
% Página de rosto (SÓ PARA A VERSÃO CORRIGIDA - APÓS DEFESA)
% Resolução CoPGr 5890 (20/12/2010)
%
% Nota: O título para as dissertações/teses do IME-USP devem caber em um
% orifício de 10,7cm de largura x 6,0cm de altura que há na capa fornecida pela SPG.
%
% IMPORTANTE:
%   Coloque um '%' em todas as linhas desta
%   página antes de compilar a versão do trabalho que será entregue
%   à Comissão Julgadora antes da defesa
%
%
%\newpage
%\thispagestyle{empty}
%    \begin{center}
%        \vspace*{2.3 cm}
%        \textbf{\Large{Título do trabalho a ser apresentado à \\
%        CPG para a dissertação/tese}}\\
%        \vspace*{2 cm}
%    \end{center}
%
%    \vskip 2cm
%
%    \begin{flushright}
%	Esta versão da dissertação/tese contém as correções e alterações sugeridas\\
%	pela Comissão Julgadora durante a defesa da versão original do trabalho,\\
%	realizada em 14/12/2010. Uma cópia da versão original está disponível no\\
%	Instituto de Matemática e Estatística da Universidade de São Paulo.
%
%    \vskip 2cm
%
%    \end{flushright}
%    \vskip 4.2cm
%
%    \begin{quote}
%    \noindent Comissão Julgadora:
%
%    \begin{itemize}
%		\item Profª. Drª. Nome Completo (orientadora) - IME-USP [sem ponto final]
%		\item Prof. Dr. Nome Completo - IME-USP [sem ponto final]
%		\item Prof. Dr. Nome Completo - IMPA [sem ponto final]
%    \end{itemize}
%
%    \end{quote}
%\pagebreak
%
%
%\pagenumbering{roman}     % começamos a numerar
%
% ---------------------------------------------------------------------------- %
% Agradecimentos:
% Se o candidato não quer fazer agradecimentos, deve simplesmente eliminar esta página
%\chapter*{Agradecimentos}
%Texto texto texto texto texto texto texto texto texto texto texto texto texto
%texto texto texto texto texto texto texto texto texto texto texto texto texto
%texto texto texto texto texto texto texto texto texto texto texto texto texto
%texto texto texto texto. Texto opcional.


% ---------------------------------------------------------------------------- %
% Resumo
\chapter*{Resumo}

\noindent ARAUJO, T. H. D. \textbf{\mydocumenttitle{}}.
2016. 20 f. Exame de qualificação (Mestrado) - Instituto de Matemática e Estatística,
Universidade de São Paulo, São Paulo, 2016.
\\

A comunidade científica pode ser enxergada como uma rede em que cada pesquisador se relaciona com outros através de colaborações diversas como, por exemplo, na coautoria de um artigo científico. Alguns trabalhos aplicam técnicas de aprendizado de máquina para prever ligações entre os participantes de uma rede, tratando do problema conhecido como Predição de Ligações. Entretanto, algumas dessas metodologias apresentam certas limitações por utilizarem formas pouco expressivas de representação do domínio, ou por analisarem apenas a estrutura da rede, sem levar em consideração as características intrínsecas dos participantes dessa rede. A proposta do presente trabalho é modelar uma ontologia capaz de indicar características próprias dos pesquisadores da Plataforma Lattes, e suas relações com outros pesquisadores, extraindo conhecimento prévio sobre o domínio e, posteriormente, utilizá-lo no enriquecimento de um modelo de aprendizado que faça predição de novas colaborações. Esperamos que esta abordagem aqui apresentada seja mais eficaz, e que contribua para uma melhor colaboração entre pesquisadores.
\\

\noindent \textbf{Palavras-chave:} redes de colaboração científica, ontologia, aprendizado de máquina, predição de ligaçoes, plataforma lattes.

% ---------------------------------------------------------------------------- %
% Abstract
\chapter*{Abstract}
\noindent ARAUJO, T. H. D. \textbf{\mydocumenttitle}.
2016. 20 f.
Exame de qualificação (Mestrado) - Instituto de Matemática e Estatística,
Universidade de São Paulo, São Paulo, 2016.
\\

The scientific community can be seen as a network of people whose relationships are built by means of various types of collaborations, like the co-authoring of a scientific paper. Some applications of machine learning to the problem of link prediction have been done before, but these methods have some limitations, because they employ representations of the domain that are not very expressive. And some of these works only analyse the structure of these networks, without taking into consideration the characteristics and attributes of the people that integrate these networks. Our proposal is to create an ontology able to show the characteristics of researchers and their relationships with others, using information from Plataforma Lattes, and extract background-knowledge about this domain that will be used to enrich a machine learning model capable of predicting new collaborations and co-authorships. We hope that our methodology will be more efficient and that it could help researchers form better relationships and collaborations.
\\

\noindent \textbf{Keywords:} scientific collaboration networks, ontology, machine learning, link prediction, plataforma lattes.

% ---------------------------------------------------------------------------- %
% Sumário
\tableofcontents    % imprime o sumário

% ---------------------------------------------------------------------------- %
\chapter{Lista de Abreviaturas}
\begin{tabular}{ll}
        SPARQL      & (\textit{SPARQL Protocol and RDF Query Language})\\
        SQL         & (\textit{Structured Query Language})\\
        SRL         & (\textit{Statistical Relational Learning})\\
        SVM         & (\textit{Support Vector Machine})\\
        SWRL        & (\textit{Semantic Web Rule Language})\\
\end{tabular}

% ---------------------------------------------------------------------------- %
%\chapter{Lista de Símbolos}
%\begin{tabular}{ll}
%        $\omega$    & Frequência angular\\
%        $\psi$      & Função de análise \emph{wavelet}\\
%        $\Psi$      & Transformada de Fourier de $\psi$\\
%\end{tabular}

% ---------------------------------------------------------------------------- %
% Listas de figuras e tabelas criadas automaticamente
\listoffigures
\listoftables

% ---------------------------------------------------------------------------- %
% Capítulos do trabalho
\mainmatter{}

% cabeçalho para as páginas de todos os capítulos
\fancyhead[RE,LO]{\thesection}

\singlespacing{}
              % espaçamento simples
%\onehalfspacing            % espaçamento um e meio

\input cap-introducao        % associado ao arquivo: 'cap-introducao.tex'
\input cap-conceitos         % associado ao arquivo: 'cap-conceitos.tex'
\input cap-correlatos        % associado ao arquivo: 'cap-correlatos.tex'
\input cap-conclusoes        % associado ao arquivo: 'cap-conclusoes.tex'
\input cap-cronograma        % associado ao arquivo: 'cap-cronograma.tex'

% cabeçalho para os apêndices
\renewcommand{\chaptermark}[1]{\markboth{\MakeUppercase{\appendixname\ \thechapter}} {\MakeUppercase{#1}} }
\fancyhead[RE,LO]{}
\appendix

% ---------------------------------------------------------------------------- %
% Bibliografia
\backmatter{} \singlespacing{}   % espaçamento simples
\bibliographystyle{plainnat-ime} % citação bibliográfica textual
\bibliography{bibliografia}  % associado ao arquivo: 'bibliografia.bib'

% ---------------------------------------------------------------------------- %
% Índice remissivo
%\index{TBP|see{periodicidade região codificante}}
%\index{DSP|see{processamento digital de sinais}}
%\index{STFT|see{transformada de Fourier de tempo reduzido}}
%\index{DFT|see{transformada discreta de Fourier}}
%\index{Fourier!transformada|see{transformada de Fourier}}

%\printindex   % imprime o índice remissivo no documento

\end{document}
