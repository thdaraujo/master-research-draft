%% ------------------------------------------------------------------------- %%
\chapter{Conceitos}
\label{cap:conceitos}

Alguns conceitos teóricos são importantes para o entendimento do presente projeto. Aqui, discutimos os fundamentos mais importantes.

%% ------------------------------------------------------------------------- %%
\section{Aprendizado Supervisionado}
\label{sec:fundamentos}

O aprendizado supervisionado é um tipo de aprendizado de máquina em que o agente, durante a fase de treinamento, observa alguns exemplos de entradas (\textit{inputs}) e de saídas (\textit{outputs}) esperadas e aprende a classificá-las, através de uma função de transformação de entradas em saídas. Depois disso, esse agente observa uma base de testes com inúmeros novos exemplos e deve classificá-los de acordo com o seu modelo interno. Dizemos que um modelo é eficaz se maximiza o conjunto de observações classificadas corretamente e minimiza as observações classificadas incorretamente.

%TODO Segundo \citet{AIMA}... ?
%SVM...

%% ------------------------------------------------------------------------- %%
\section{Conhecimento prévio (\textit{background knowledge})}
\label{sec:fundamentos}

Conhecimento prévio, segundo \citet[capítulo 19]{AIMA}, geralmente é representado como um conjunto geral de teorias em uma lógica de primeira ordem. Essas teorias são compostas por hipóteses que devem explicar ou classificar corretamente um conjunto de observações (ou atributos). Essas observações são sentenças lógicas que descrevem algo sobre o mundo. As hipóteses devem poder ser generalizadas e  aplicáveis a novos exemplos. Outra propriedade importante é que a teoria seja consistente, ou seja, uma hipótese não pode gerar falsos positivos ou falsos negativos.

Esse conhecimento prévio pode ser cumulativo: novas observações podem gerar novas hipóteses, que enriquecem o conhecimento prévio e o modelo como um todo, tornando mais eficaz a sua capacidade de predição e aumentando a generalidade de suas hipóteses.

Uma das disciplinas que se concentra nesse tipo de problema de aprendizado é chamada de \textit{Inductive Logic Programming} (ILP). Problemas expressos de forma relacional podem ser tratados muito bem com algoritmos de ILP. Com o uso de conhecimento prévio, segundo \citet{AIMA}, é reduzida a complexidade do aprendizado, pois as novas hipóteses geradas devem ser consistentes com as hipóteses do conhecimento prévio, e isso reduz o conjunto de novas hipóteses possíveis que o algoritmo precisaria considerar, além de conseguir explicar parte das novas observações, através do uso das hipóteses já integradas ao conhecimento prévio.

%% ------------------------------------------------------------------------- %%
\section{Hierarquia de Representações em Problemas de Aprendizado}
\label{sec:fundamentos}

Segundo \citet{Raedt2008}, existem várias representações que podem ser utilizadas para modelar um problema de aprendizado, algumas mais expressivas, outras menos. Ele mostra em seu trabalho que também existe uma hierarquia dessas representações em uma ordem crescente de expressividade, exibida a seguir:

\begin{itemize}
    \item \textbf{Representações Booleanas} (\textit{Boolean Learning}): cada percepção contém ítens ou proposições verdadeiras (ou \textit{presentes}) e falsas (ou \textit{não-presentes}), e desejamos encontrar uma regra que defina essas observações. Esta é a representação com o menor grau de expressividade.
    \item \textbf{Aprendizado por tabelas de valores e atributos} (\textit{Attribute-Value Learning}): o conjunto de percepções ou experiências é apresentado como uma tabela única onde cada linha é um exemplo e cada coluna é um atributo, e desejamos que o modelo aprenda a classificar uma experiência nova ou a prever o valor de um atributo de acordo com os exemplos vistos anteriormente durante o treinamento. Esse tipo de representação é bastante utilizada e bem comum.
    \item \textbf{Representações Multi-Instância} (\textit{Multi-Instance Representations}): muito parecida com a representação por valores e atributos, só que, neste caso, classes podem conter múltiplos exemplos, ou seja, um exemplo pode ter um atributo que depende do valor contido em outro exemplo.
    \item \textbf{Aprendizado Relacional} (\textit{Relational Learning}): múltiplos exemplos ou relações entre exemplos e hipóteses podem aparecer, e é possível construir essa representação com múltiplas tabelas em um banco de dados. É bastante útil quando o domínio possui uma natureza relacional.
    \item \textbf{Programas Lógicos}  (\textit{Logical Programs}): é um modelo que recebe como entrada as diversas observações (exemplos), e aprende a sintetizar um programa lógico com regras gerais (hipóteses) capazes de derivar estes exemplos e classificar outros conjuntos de observações. Existem alguns trabalhos na literatura cuja linguagem de representação foi PROLOG, e que usavam ferramentas de Programação Lógica Indutiva (\textit{Inductive Logic Programming}). Tal representação é a mais expressiva.
\end{itemize}

Apesar dessa hierarquia de expressividade, Raedt nos mostra que as representações são equivalentes em sua capacidade de representação: o que pode ser modelado por uma pode ser adaptado e transformado em outra representação de mais baixo nível, e vice-versa. Entretanto, um modelo de mais alto nível é declarativo e compacto. Quando transformado em um modelo de nível mais baixo, necessita de um número muito maior de exemplos, atributos, tabelas, colunas e outras entidades. Isso dificulta a análise e a apreciação dessas informações. Portanto, a escolha correta da representação é fundamental para o pesquisador que deseja explorar um problema, pois a expressividade da representação simplifica a análise do domínio.

Raedt também conclui que todas essas representações podem ser modeladas naturalmente com lógica de predicados, que é relacional por natureza. Também dá exemplos de vários algoritmos de aprendizado utilizando programação com lógica,
como \textit{SVMs}, \textit{k-nearest neighbor} e \textit{redes bayesianas}, que podem ser utilizados na modelagem do problema.

%% ------------------------------------------------------------------------- %%
\section{Ontologia}
\label{sec:ontologia}

Diferentemente da noção filosófica de Ontologia como o estudo da natureza do Ser, para a Ciência da Computação uma ontologia é uma estrutura relacional, um modelo de especificação formal explícita de conceitos de um determinado domínio.
Esses conceitos podem dizer respeito a entidades, ou a relações entre entidades desse mesmo domínio.

Segundo \citet{Guarino2009}, uma ontologia é um artefato computacional que possui um modelo formal de representação da estrutura de um domínio, contendo as entidades relevantes bem como as relações que emergem da observação desse mesmo domínio, mas apenas as que são úteis para algum determinado fim. Geralmente, desejamos responder a algumas questões de competência. A modelagem de classes e relações entre entidades é feita através de definições formais em uma lógica de descrição, de forma compacta e de alto nível de abstração.

Um exemplo de domínio possível é a comunidade científica, com seus diversos pesquisadores e suas relações com outros pesquisadores. Uma ontologia que representa esse domínio deve ser capaz de capturar as entidades relevantes, organizando as informações em conceitos e relações, para que seja capaz de responder a algumas perguntas, como \textit{Quem é coautor de um artigo A?}. Nesse caso, pode descrever classes de entidades como \textit{Pesquisador} e \textit{Artigo}, e relações binárias entre dois indivídios, como a relação \textit{colaborou\_com} e \textit{publicou\_artigo}.

O mais importante é que essa definição formal possa ser entendida por um computador em um formato padronizado, para que a resposta seja encontrada de forma automática. Para isso, uma ontologia pode utilizar a inferência lógica para responder a esse tipo de pergunta, como também descobrir novas relações entre entidades, através do uso de \textit{reasoners}, que são programas capazes de inferir conseqüências lógicas a partir de uma base de fatos contidos em uma ontologia.

Por tudo isso, podemos dizer que uma ontologia é uma representação bastante expressiva e poderosa.

%% ------------------------------------------------------------------------- %%
\section{OWL}
\label{sec:owl}

%TODO owl
