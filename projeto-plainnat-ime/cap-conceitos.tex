%% ------------------------------------------------------------------------- %%
\chapter{Conceitos}
\label{cap:conceitos}

Alguns conceitos teóricos são importantes para o entendimento do presente projeto. Aqui, discutimos os fundamentos mais importantes.

%% ------------------------------------------------------------------------- %%
\section{Aprendizado Supervisionado}
\label{sec:fundamentos}

O aprendizado supervisionado é um tipo de aprendizado de máquina em que o agente, durante a fase de treinamento, observa alguns exemplos de entradas (\textit{inputs}) e de saídas (\textit{outputs}) esperadas e aprende a classificá-las, através de uma função de transformação de entradas em saídas. Depois disso, esse agente observa uma base de testes com inúmeros exemplos e classifica esses dados de acordo com o seu modelo interno.



Segundo \citet{AIMA}...


%TODO Chapter 19 do AIMA fala sobre background knowledge

%SVM...


%% ------------------------------------------------------------------------- %%
\section{Ontologia}
\label{sec:ontologia}


%% ------------------------------------------------------------------------- %%
\section{OWL}
\label{sec:owl}


%% ------------------------------------------------------------------------- %%
\section{RDF}
\label{sec:rdf}
