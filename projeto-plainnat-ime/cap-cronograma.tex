%% ------------------------------------------------------------------------- %%
\chapter{Cronograma}
\label{cap:cronograma}

O cronograma de atividades está descrito na tabela a seguir.

\begin{table}[h!]
    \centering
    \begin{tabular}{|c|c|c|c|c|c|c|c|c|c|}
     \hline
       & Mar & Abr & Mai & Jun & Jul & Ago & Set & Out & Nov \\
     \hline\hline
     1 & X &   &   &   &   &   &   &   &   \\
     \hline
     2 & X & X &   &   &   &   &   &   &   \\
     \hline
     3 &   & X & X &   &   &   &   &   &   \\
     \hline
     4 &   &   & X & X &   &   &   &   &   \\
     \hline
     5 &   &   &   &   & X & X & X &   &   \\
     \hline
     6 &   &   &   &   & X & X & X & X &   \\
     \hline
     7 &   &   &   &   &   &   &   &   & X \\
     \hline
    \end{tabular}
    \caption{Cronograma de atividades}
    \label{cronograma-atividades}
\end{table}

\begin{itemize}
    \item [\textit{Item 1.}] Estudar o domínio e modelar uma Ontologia.
    \item [\textit{Item 2.}] Modelar as consultas à ontologia que servirão para enriquecer o modelo de predição.
    \item [\textit{Item 3.}] Construir um modelo de predição de ligações baseado nos trabalhos da literatura, fazendo experimentos com diferentes técnicas de aprendizado supervisionado.
    \item [\textit{Item 4.}] Propor e implementar um algoritmo de extração de conhecimento da ontologia e enriquecimento do modelo de predição.
    \item [\textit{Item 5.}] Montar ambiente de testes e elaborar alguns testes de comparação.
    \item [\textit{Item 6.}] Analisar os resultados e escrever a dissertação.
    \item [\textit{Item 7.}] Defender a dissertação.

\end{itemize}
