%% ------------------------------------------------------------------------- %%
\chapter{Conclusões}
\label{cap:conclusoes}

Uma ontologia é uma forma mais expressiva de representação de conhecimento, e com ela, podemos gerar conhecimento expressivo e de alto-nível sobre o domínio estudado. O uso dessa ferramenta no estudo das redes de colaboração trará uma melhor utilização do conhecimento e uma melhor exploração do domínio, o que pode auxiliar na análise e descoberta de novas características dos pesquisadores que fazem parte dessas redes, assim como na geração e avaliação de novas relações entre esses pesquisadores, publicações e demais entidades.

Esperamos, com o modelo proposto no presente trabalho, obter um aumento da eficácia de modelos de predição de ligações, através do uso desse conhecimento extraído da ontologia. Com isso, desejamos avançar o estado-da-arte no tratamento do problema de predição de ligações, e construir um modelo que possa ser utilizado por outros que se interessem por esse tipo de problema.
