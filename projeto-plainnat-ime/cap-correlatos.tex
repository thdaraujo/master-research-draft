%% ------------------------------------------------------------------------- %%
\chapter{Trabalhos Correlatos}
\label{cap:correlatos}

Neste capítulo, apresentamos algumas contribuições importantes que tratam do problema de análise de redes de colaboração científica e predição de ligações, e outros trabalhos relacionados que serviram de inspiração para este projeto de pesquisa.


%% ------------------------------------------------------------------------- %%
\section{scriptLattes}
\label{sec:scriptLattes}

A Plataforma Lattes é uma plataforma vinculada ao CNPq, e a mais importante base integrada de currículos, grupos de pesquisa e instituições de ensino do Brasil, registrando informações valiosas sobre as atividades de pesquisa e o perfil de pesquisadores de diversas áreas do Saber em todo o país.

O projeto \textit{scriptLattes}, proposto por \citet{Mena-Chalco2009}, é um sistema capaz de fazer mineração dos dados de currículos presentes na Plataforma Lattes e de gerar vários relatórios acadêmicos, além de disponibilizar informações sobre as publicações dos pesquisadores brasileiros, fazendo desambiguação dos autores e artigos e exportando dados sobre coautoria e outros tipos de colaboração.

O nosso trabalho vai utilizar uma base de mais de 4 milhões de currículos extraídos da Plataforma Lattes, gentilmente cedida pelo professor Jesús P. Mena-Chalco, como base de testes para o modelo aqui proposto.


%% ------------------------------------------------------------------------- %%
\section{Predição de Ligações}
\label{sec:link-prediction}

Vários trabalhos presentes na literatura exploram o problema da predição de ligações (\textit{link prediction}) em redes sociais. Esse problema possui diversas aplicações, como na análise e reconstrução de redes e em sistemas que utilizam informações pessoais para sugerir novos contatos ou novos amigos. Outros pretendem detectar membros de redes terroristas com o intuito de prevenir ataques.

Uma outra aplicação interessante, discutida no item ~\ref{ssec:srl}, conseguia encontrar artigos ou documentos relacionados e sugerir citações.

Discutimos a seguir alguns trabalhos relevantes.

%% ------------------------------------------------------------------------- %%
\subsection{Importância de um pesquisador, financiamento, e quantidade de colaboradores em um projeto}
\label{ssec:importance}

Em \citet{Ebadi2015}, encontramos uma análise muito apurada sobre uma rede de colaboração científica que serviu para levantar os fatores que determinam a importância de um pesquisador nessa rede, sua centralidade e sua produtividade. Foram feitas várias constatações interessantes. Descobriram, por exemplo, que grupos ligados a organizações que produziam muitos artigos importantes acabavam tendo uma performance melhor do que outros grupos.

Outro fato interessante é que a experiência e os anos de trabalho de um indivíduo fazem com que ele seja mais conhecido em sua comunidade. Tendo acesso a dados sobre financiamento e fomento à pesquisa, constataram que pesquisas fomentadas por indústrias e empresas acabam atraindo mais colaboradores.

Essa análise também foi baseada em medidas feitas na estrutura da rede, como a centralidade dos vértices, \textit{eigenvectors} e coeficientes de agrupamento, numero médio de coautores (ligações) de um dado pesquisador, dentre outros indicadores. Com essas medidas, foi possível encontrar os líderes de diversas comunidades, que são pessoas com grande influencia local.

Descobriram, ainda, que os pesquisadores mais produtivos e com o trabalho de melhor qualidade também eram os mais colaborativos. Também mostraram a influência que o financiamento causa nas redes locais de colaboração ao longo do tempo, fazendo com que os cientistas que recebem financiamento e ocupam posições de liderança busquem coautores em outras comunidades mais distantes, ampliando o seu canal de conexões.

%TODO citar hiperautoria? Tem relacao com colaboracoes... Poderia citar rapidamente um daqueles trabalhos sobre hiperautoria.


%% ------------------------------------------------------------------------- %%
\subsection{Predição utilizando Aprendizado Supervisionado}
\label{ssec:graphs-with-attributes}

\citet{MohammadAlHasan} aplicou e comparou a eficácia de diversas técnicas de aprendizado supervisionado, tais como \textit{SVMs}, \textit{Árvores de Decisão}, \textit{Multilayer Perceptrons}, modelos de classificação utilizando kNN (\textit{k-nearest neighbors}), \textit{Naive Bayes}, e \textit{RBF Networks}. Sua aplicação foi na detecção de membros de redes terroristas.

O uso de \textit{SVMs} mostrou-se melhor do que as outras técnicas, por obter uma taxa de acerto um pouco maior e por possibilitar a escolha das características importantes a serem utilizadas pelo modelo através de um método automatizado de filtragem.

O aspecto mais importante desse projeto é que a escolha do algoritmo não parece fazer tanta diferença assim na eficácia do modelo. Podemos, com isso, concluir que uma boa escolha das características é um dos fatores mais importantes.


%% ------------------------------------------------------------------------- %%
\subsection{Grafos Relacionais com Atributos}
\label{ssec:graphs-with-attributes}

Um experimento muito interessante e inspirador foi explorado por \citet{Cervantes2014}, que modelou a rede de colaboração científica através de grafos relacionais com atributos para mostrar as relações de coautoria entre pesquisadores. Foi também criado um modelo baseado em \textit{SVMs} capaz de prever novas colaborações (ligações) entre pesquisadores a partir de dados de treinamento extraídas da Plataforma Lattes via \textit{scriptLattes}.

O modelo proposto permitiu uma série de outras análises relevantes da estrutura dessa rede, como a identificação de pesquisadores mais importantes, ou mais colaborativos, que correspondem aos vértices com mais conexões. Também foi possível identificar comunidades dentro de diferentes áreas formadas por componentes conexos desse grafo.

Por sua flexibilidade e alto nível de expressividade, esse modelo em grafo com atributos pode ser facilmente expandido. Entretanto, o modelo de aprendizado utilizado recebe como entrada tabelas de atributos e valores. Por causa disso, é preciso transformar uma representação relacional mais expressiva (um grafo) em uma representação mais simples (uma tabela) para que o algoritmo funcione.

%% ------------------------------------------------------------------------- %%
\subsection{Predição de Citações}
\label{ssec:srl}

Uma aplicação de Aprendizado Estatístico Relacional (SRL) foi feita por \citet{Popescul2003} em um sistema de predição de citações, que nada mais é do que um modelo de predição de ligações entre documentos. Analisando informações a respeito de duas publicações, o sistema calculava a probabilidade de uma publicação citar a outra. Com isso, o sistema poderia sugerir trabalhos correlatos que pudessem ser citados por um autor durante a escrita de algum artigo ou outro documento.

O grande desafio desse modelo de aprendizado é a seleção de atributos (\textit{features}) que auxiliem na tarefa de classificação. Como o problema está modelado de forma relacional, a quantidade de atributos pode crescer indefinidamente. É preciso, portanto, encontrar uma forma inteligente de adicionar um atributo e avaliar sua relevância, através de algum tipo de verificação.

Em um trabalho posterior, \citet{Popescul2007} sugerem um modelo interessante de seleção de atributos: a geração de novos atributos se dá via busca no conjunto de atributos possíveis, e sua seleção é feita através de um teste que avalia a significância estatística da inclusão desse atributo no modelo de predição.

Aplicado ao mesmo problema de predição de citações, o conjunto de atributos possíveis é derivado de consultas SQL relacionadas às entidades pertencentes a um banco de dados relacional, e o algoritmo explora esse conjunto e seleciona os atributos mais relevantes segundo sua significância estatística. O interessante desse trabalho é que o algoritmo, ao gerar novos atributos, adiciona essas novas informações ao mesmo banco de dados como novas relações entre entidades, enriquecendo ainda mais o modelo.

Nossa proposta é parecida, pois também desejamos gerar novos atributos para essas entidades, explorando algumas consultas possíveis, extraindo conhecimento e formalizando esse resultado como \textit{background-knowledge}, que será usado para enriquecer o modelo de aprendizado.  A principal diferença está no uso de uma ontologia em vez de um banco de dados relacional.


%% ------------------------------------------------------------------------- %%
\section{Ontologias}
\label{sec:ontologias}

Uma rede social é uma estrutura relacional que podemos facilmente representar por um grafo ou por uma ontologia. Alguns trabalhos modelaram ontologias para tratar de problemas relacionados à Plataforma Lattes e outras redes de colaboração, que discutimos a seguir.


%% ------------------------------------------------------------------------- %%
\subsection{Plataforma Lattes}
\label{ssec:lattes}

No caso da Plataforma Lattes, \citet{Anaue2009} construíram uma ontologia capaz de responder a diversas questões de competência. O usuário consulta a base usando linguagem natural, e a ontologia consegue encontrar a informação desejada, através de inferência. A vantagem dessa metodologia é a facilidade na modelagem, nas ferramentas de inferência, e na
expressividade da representação. \citet{Galego2013} também desenvolveu uma ontologia para esse domínio como uma extensão do trabalho anterior, e seu interesse foi gerar relatórios e detectar inconsistências.

%% ------------------------------------------------------------------------- %%
% TODO pensar em um nome para essa seção
\section{Bancos de Dados}
\label{sec:bd}

%% ------------------------------------------------------------------------- %%
\subsection{MongoDB}
\label{ssec:mongodb}

Um recente trabalho de [\citenum{Botoeva}] utiliza NoSQL em aplicações OBDA,
propondo uma arquitetura genérica de OBDA aplicado a qualquer tipo de SGBD.
A primeira aplicação desta arquitetura foi feita com o MongoDB. Com isso,
foi possível mapear uma ontologia com um conjunto salvo em um banco de dados
e fazer a conversão da consulta SPARQL em uma consulta MongoDB, eliminando
assim a necessidade de se extrair previamente esses dados e transformá-los
em RDF ou popular a ontologia antes da consulta.

[\citenum{michel2016mapping}] construiu uma ferramenta de mapeamento de
consultas SPARQL em consultas a uma base MongoDB para tornar públicos os dados
de uma base legada.

A técnica utilizada consiste em gerar uma base RDF virtual, ou seja, os
documentos salvos na base foram expostos em formato RDF. Foi proposto um
método de tradução em 2 passos: primeiro, a consulta SPARQL é transformada em
uma consulta abstrata usando mapeamentos MongoDB para RDF escritos em uma
linguagem intermediária chamada xR2RML. Depois, essa consulta intermediária
é transformada em uma consulta MongoDB concreta. Como resultado,
concluíram que é sempre possível reescrever uma consulta que produza resultados
corretos.


%% ------------------------------------------------------------------------- %%
\section{Justificativa}
\label{sec:justificativa}

A principal hipótese a ser testada é se o enriquecimento de um modelo de aprendizado supervisionado com conhecimento prévio extraído de uma ontologia pode aumentar a eficácia desse modelo quando aplicado ao problema de predição de ligações, aqui definido como a colaboração entre pesquisadores em uma rede de colaboração. Essa proposta será comparada com os trabalhos de \citet{MohammadAlHasan} e \cite{Cervantes2014}.

O uso de uma ontologia é justificado por sua expressividade, algo que simplifica a criação de consultas geradoras de novos atributos para as entidades da rede, e a possibilidade da descoberta de novas características, através do uso de inferência lógica. Também é simples encontrar e representar novos tipos de relações entre entidades, bem como organizá-las em diferentes classes. Outra vantagem secundária é a uniformização do vocabulário a respeito do domínio, e a fácil reutilização desses conceitos e conhecimentos em outras aplicações de domínios semelhantes.

Além disso, os resultados do modelo de predição podem ser aproveitados posteriormente para expandir a própria ontologia inicial. As predições resultantes do modelo podem ser adicionadas à ontologia como novas relações. E isto pode gerar novos atributos úteis e novos tipos de classificação que podem ser extraídos novamente, derivando um conhecimento prévio ainda mais rico.

Escolhemos aplicar esse modelo à Plataforma Lattes por ser um domínio bem utilizado e por termos disponível uma base de currículos bastante extensa, da ordem de 4 milhões de arquivos.

Quanto ao modelo de aprendizado supervisionado, serão utilizados os mesmos que estão presentes na literatura para que se faça uma justa comparação.
