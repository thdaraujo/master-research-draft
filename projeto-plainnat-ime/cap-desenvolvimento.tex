%% ------------------------------------------------------------------------- %%
\chapter{Desenvolvimento}
\label{cap:desenvolvimento}

Neste capítulo, apresentamos o desenvolvimento do projeto e as principais práticas que o norteiam, através da discussão de um experimento exploratório em pequena escala (com uma quantidade pequena de dados) que servirá como estrutura inicial do projeto.

%% ------------------------------------------------------------------------- %%
\section{Estudo do domínio e construção de uma ontologia a respeito da Rede de Colaboração Científica}
\label{sec:desenvolvimento-ontologia}

A proposta consiste em rodar um experimento em pequena escala. Extraímos informações de um número pequeno de currículos Lattes, com interesse nas informações básicas dos pesquisadores do Departamento de Computação do Instituto de Matemática e Estatística da USP. Dados como nome, área de atuação, publicações e coautores, e local de trabalho ou residência.

Com base nessas informações, modelamos uma ontologia utilizando OWL através do software Protégé. A estrutura básica da ontologia serve para organizar o conhecimento a respeito desses pesquisadores e sua origem acadêmica, suas publicações, e os relacionamentos que eles têm com outros. Com isso, podemos consultar essas informações estruturadas acerca da produção bibliográfica e científica dos docentes e alunos através de queries SPARQL. Essas queries geram um conjunto de dados que serve de base para o experimento.

Partimos da ontologia Friend of a friend (FOAF), que descreve pessoas, relações e suas atividades, por ela ter alguns conceitos, classes e termos que nos auxiliam na modelagem da rede de colaboração científica.

A ontologia possui a seguinte estrutura de classes (TBOX):

\begin{alltt}
Class: Grupo \( \equiv \) foaf:Group
SubClassOf: Agent
Descrição: \emph{Classe que define grupos de pessoas.}
Relações:
  membros \( \equiv \) member

Class: Organização \( \equiv \) foaf:Organization
SubClassOf: Agent
Descrição: \emph{Define uma organização, que pode ser uma Universidade, por
exemplo.}
Relações:
  faz_parte \( \equiv \) sub_organizacao: \emph{Uma \textbf{Organização}
pode fazer parte de outra \textbf{Organização}, por exemplo: o
Instituto de Física faz parte da USP.}
  possui \( \equiv \) super_organizacao: \emph{Uma \textbf{Organização} pode
possuir outra \textbf{Organização}, por exemplo: A USP possui o
Instituto de Física.}

Class: Universidade
SubClassOf: Organização
Descrição: \emph{Classe das universidades e instituições
de ensino.}

Class:Instituto
SubClassOf: Organização
Descrição: \emph{Classe dos institutos de ensino e pesquisa e faculdades.}

Class: Departamento
SubClassOf: Organização
Descrição: \emph{Classe dos departamentos ligados a institutos de pesquisa ou
faculdades.}

Class: Revista
SubClassOf: Organização
Descrição: \emph{Classe das revistas científicas que publicam artigos.}
Relações:
  publicou: \emph{Uma \textbf{Revista} pode publicar um \textbf{Artigo}. É o
  inverso da relação \textbf{publicado_em}.}

Class: Pessoa \( \equiv \) foaf:Person
SubClassOf: Agent
Descrição: \emph{Classe-pai dos tipos de pessoa.}
Relações:
  cursou: \emph{Uma \textbf{Pessoa} pode cursar algum \textbf{Curso}.}
  estudou_em: \emph{Uma \textbf{Pessoa} pode ter estudado em alguma
\textbf{Universidade, Instituto, Faculdade}.}
  membro_de: \emph{uma \textbf{Pessoa} pode ser membro de um  \textbf{Grupo}.
  É o inverso da relação \textbf{membro}.}
  autor: \emph{Uma \textbf{Pessoa} pode ser autora de um  \textbf{Documento}.
  É o inverso da relação \textbf{tem_autores}.}
  trabalhou_em: \emph{uma \textbf{Pessoa} pode trabalhar ou ter trabalhado
  em uma \textbf{Organização}.}

Class: Aluno
SubClassOf: Pessoa
Descrição: \emph{Classe dos indivíduos que são alunos de algum curso.}
Relações:
  tem_orientador: \emph{Um \textbf{Aluno} pode ser orientado por um
  \textbf{Professor} ou outro pesquisador.}

Class: Professor
SubClassOf: Pessoa
Descrição: \emph{Classe dos professores.}
Relações:
  orienta: \emph{Um \textbf{Professor} pode orientar um
  \textbf{Aluno}. É o inverso da relação \textbf{tem_orientador}.}
  tem_orientador: \emph{Um \textbf{Professor} pode ser orientado por
  um outro \textbf{Professor}, mesmo que no passado.}

Class: Curso
SubClassOf: Thing
Descrição: \emph{Classe dos cursos de graduação e pós-graduação.}
Relações:
  cursado_por: \emph{Um \textbf{Curso} pode ser cursado por \textbf{Pessoas}.}
  tipo_de_curso: \emph{Um \textbf{Curso} pode ser de Pós-Graduação ou
  Graduação.}
  oferecido_por: \emph{Pode ser oferecido por uma instituição.}

Class: Graduação
SubClassOf: Tipo de Curso
Descrição: \emph{Classe dos cursos de graduação.}

Class: Pós Graduação
SubClassOf: Tipo de Curso
Descrição: \emph{Classe dos cursos de pós-graduação.}

Class: Documento \( \equiv \) foaf:Document
SubClassOf: Thing
Descrição: \emph{Classe dos documentos produzidos por alunos,
professores e pesquisadores, como um artigo científico.}
Relações:
  tem_autores: \emph{Um documento possui um ou mais autores do tipo
  \textbf{Pessoa}. Essa propriedade é importante pois indica uma relação de
  \textbf{coautoria} entre dois pesquisadores.}

Class: Publicacao
SubClassOf: Documento
Descrição: \emph{Classe dos artigos científicos e outras publicações de
revistas, simpósios e outros eventos.}
Relações:
  publicado_em: \emph{Uma \textbf{Publicação} pode ser publicada em
  uma \textbf{Revista} ou um \textbf{Evento}.}

Class: Teseg
SubClassOf: Documento
Descrição: \emph{Classe-pai das Teses científicas de doutorado e
dissertações de mestrado.}

Class: Eventos
SubClassOf: Thing
Descrição: \emph{Classe-pai dos eventos científicos, conferências, e simpósios.}
Relações:
  publicou: \emph{Um \textbf{Evento} pode ter publicações de \textbf{Artigos}.}

Class: foaf:Spatial Thing
Descricão: \emph{Classe das entidades que possuem alguma
localização geográfica. \textbf{Evento} é uma delas.}

Classe: Local
SubClassOf: Thing
Descrição: \emph{Classe dos locais geográficos.}
Relações:
  localizado \( \equiv \) foaf:based near \emph{Relaciona \textbf{Organização}
  ou \textbf{Evento} com um local geográfico.}

Class: País
SubClassOf: Local

Class: Continente
SubClassOf: Local

Class: Cidade
SubClassOf: Local

\end{alltt}

%% ------------------------------------------------------------------------- %%
\section{Consultas}
\label{sec:desenvolvimento-consultas}

A ontologia foi populada com as informações dos currículos lattes escolhidos, e algumas consultas também fizeram parte da definição da ontologia. Essas consultas foram feitas na linguagem SPARQL. As consultas que nos interessam agora respondem às seguintes questões:

\begin{itemize}
    \item Qual a área de pesquisa de um pesquisador?
    \item Quem são os pesquisadores que ele orientou?
    \item Há quantos anos esse pesquisador trabalha em uma instituição?
    \item Quem são as pessoas que colaboraram, como coautores, em sua produção científica?
\end{itemize}

A primeira questão de competência é extraída diretamente do currículo lattes, quando possível. Trata-se da definição de suas áreas de atuação.

A segunda questão pode ser respondida através da relação \textit{orienta} das instâncias da classe \textit{Professor}. Um professor terá uma lista de orientandos, que também são pesquisadores. A origem dessa informação é a seção de orientações do currículo Lattes.

A terceira questão pode ser respondida através da análise da atuação profissional em instituições de ensino, pois ela contém o ano de início, informação extraída diretamente do Lattes.

Por fim, a última questão é obtida fazendo-se uma consulta aos artigos publicados pelo pesquisador, aos trabalhos em eventos e demais tipos de produção técnica e outras informações bibliográficas que listam os nomes de outros colaboradores. O grande desafio nesse caso é fazer a desambiguação de pesquisadores.

%% ------------------------------------------------------------------------- %%
\section{Predição de Ligações}
\label{sec:desenvolvimento-predicao}

%TODO ???
%Já a questão a respeito dos anos de trabalho em uma dada instituição é extraída da data em que

%e cervantes??

%% ------------------------------------------------------------------------- %%
\section{Extração do Conhecimento e Construção do \textit{background-knowledge}}
\label{sec:desenvolvimento-background-knowledge}

%TODO Conhecimento Prévio

O conhecimento prévio, que chamaremos de $BK$, será extraído da ontologia através das consultas SPARQL e estruturado em uma matriz da seguinte forma: seja $P$ o conjunto de pesquisadores presentes na ontologia. Cada linha da tabela resultante corresponde a um elemento do conjunto de pares distintos de pesquisadores $Q = \{ (a, b) | a \in P \text{ e } b \in P \text{ e } a \neq b \}$, e cada coluna recebe o valor de algum atributo de $a$, ou de $b$, ou $R(a,b)$ (uma relação entre $a$ e $b$). $R(a,b) = 1$ se a relação existir ou for válida, e $R(a,b) = -1$ em caso contrário.

Como exemplo, admita um conjunto $P = \{ p_1, p_2, p_3 \}$ e algumas relações como $R = \{ orienta, coautor \}$. Sabemos que $p_1$ \texttt{orienta} $p_2$, pois $p_1$ foi orientador de $p_2$ em seu doutorado. E também sabemos que $p_1$ \texttt{coautor} $p_2$ e $p_1$ \texttt{coautor} $p_3$ e $p_2$ \texttt{coautor} $p_3$, pois os três publicaram um artigo juntos. Também conhecemos alguns atributos desses pesquisadores: todos eles pertencem à área Ciência da Computação cujo identificador será, digamos, 123. Portanto, a matriz resultante será:

\begin{table}[h!]
    \centering
    \begin{tabular}{|c|c|c|c|}
     \hline
      & área & orienta & coautor   \\
     \hline\hline
     $Q(p_1, p_2)$ & 123 & 1  & 1  \\
     \hline
     $Q(p_1, p_3)$ & 123 & -1 & 1  \\
     \hline
     $Q(p_2, p_3)$ & 123 & -1 & 1  \\
     \hline
    \end{tabular}
    \caption{Matriz de \textit{background-knowledge} }
    \label{matriz-relacoes}
\end{table}

Entretanto, essa matriz pode ter um tamanho muito grande, dependendo da quantidade de pesquisadores. Uma solução possível seria limitar os exemplos, mostrando apenas os pares de pesquisadores que também possuem relação de coautoria. Entretanto, o impacto dessa alteração na eficácia do método ainda precisa ser investigado.

O modelo de predição receberá como entrada esse conjunto de dados $BK$, separado aleatoriamente em subconjuntos de treinamento e teste. Após o treinamento, durante os testes, o modelo deverá prever se existe ou não uma relação de coautoria a partir dos atributos desse par de pesquisadores fazendo a classificação de cada um dos exemplos. Esse resultado será posteriormente validado com o conjunto original, sendo possível com isso medir a eficiência do método.

%Enriquecimento

O conjunto de informações geradas através do grafo com atributos será enriquecido com as informações extraídas do conjunto $BK$. Chamamos aqui de enriquecimento a adição de novos atributos aos elementos do conjunto $P$. Cada pesquisador possui um conjunto de atributos relevantes, e esse conjunto será expandido com as informações do conjunto $BK$.

%??

%% ------------------------------------------------------------------------- %%
\section{Desenvolvimento e Testes}
\label{sec:desenvolvimento-testes}

A estrutura inicial do experimento segue o diagrama abaixo. Note que existem duas fases neste experimento, a fase A e a fase B, cujos passos discutimos a seguir.

%TODO explicar diagrama
\begin{figure}[!h]
  \centering
  \includegraphics[width=.99\textwidth]{diagrama-menor}
  \caption{Diagrama de execução do experimento}
  \label{fig:diagrama-experimento}
\end{figure}

%TESTES

Vamos testar inicialmente dois modelos de predição: Máquinas de Vetores de Suporte e Árvores de Decisão.
SMVs são importantes porque...
Árvores de decisão são...

O modelo receberá como estrada um conjunto de treinamento onde cada linha corresponde a um elemento do conjunto $Q$, sendo cada coluna uma propriedade de um dos pesquisadores, um atributo ou uma relação extraídos de $BK$.

Um dos modelos será treinado com dados apenas dos pesquisadores (conjunto inicial sem o enriquecimento), e será testado, tendo sua eficácia será analisada.

Outro modelo será treinado apenas com o conjunto enriquecido com as informações de $BRK$, será testado, e sua eficácia será analisada.

Por fim, faremos a comparação da eficácia entre ambos os modelos, verificando se há melhoria na eficácia quando utilizamos conhecimento prévio.

Também é de interesse avaliar se os atributos adicionais extraídos de $BK$ são relevantes. O modelo SVM é %???


O ambiente de testes recebe um conjunto de currículos e um arquivo OWL correspondente à estrutura da ontologia, porém sem nenhuma instância.

(1)O programa será responsável por ler os currículos e popular a ontologia com esses dados.

(2) Um dos módulos será responsável por construir o grafo com atributos e extrair o conjunto de dados sobre pesquisadores e o vetor de características. %pegar de cervantes
%como? quais atributos? como é o grafo?

(3) Após isso, deve rodar consultas SPARQL e extrair o conhecimento prévio $BK$.

(4) Com esses arquivos gerados, o programa vai ler os dados extraídos por (2) e enriquecer o conjunto $BK$ com o vetor de características.

(5) O programa deve instanciar um modelo de predição e triná-lo com as informações apenas dos vetores de características
(6) O programa deve instanciar outro modelo de predição e treinálo com as informações enriquecidas em $BK$

(7) O programa deve rodar os testes de ambos os modelos, calculando sua eficácia, e deve gerar um relatório comparando ambas as metodologias.

(8) O programa deve ainda avaliar se as características contidas nos atributos de $BK$ são relevantes.



%1. Estudar o domínio e modelar uma Ontologia.
%2. Modelar as consultas à ontologia que servirão para enriquecer o modelo de predição.
%3. Construir um modelo de predição de ligações baseado nos trabalhos da literatura, fazendo experimentos com diferentes técnicas de aprendizado supervisionado.
%4. Propor e implementar um algoritmo de extração de conhecimento da ontologia e enriquecimento do modelo de predição.
%5. Montar ambiente de testes e elaborar alguns testes de comparação.
