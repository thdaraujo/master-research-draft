%% ------------------------------------------------------------------------- %%
\chapter{Desenvolvimento}
\label{cap:desenvolvimento}

A proposta inicial do projeto consiste em rodar um experimento em pequena escala. Extraímos informações de um número pequeno de currículos Lattes, com interesse nas informações básicas dos pesquisadores do Departamento de Computação do Instituto de Matemática e Estatística da USP. Dados como nome, área de atuação, publicações, coautores, local de trabalho ou residência, e data em que começou a trabalhar no instituto.

Com base nessas informações, modelamos uma ontologia utilizando OWL através do software Protégé. Esse software facilita a construção, edição, e análise de ontologias.

Essa ontologia tem por finalidade organizar o conhecimento a respeito de pesquisadores e sua origem acadêmica, suas publicações, e os relacionamentos entre essas entidades. Com isso, é possível acessar informações estruturadas acerca da produção bibliográfica e científica dos docentes e alunos do instituto, servindo de base para esse experimento.

%TODO falar do FOAF

A ontologia possui a seguinte estrutura (TBOX):

%TODO conferir ontologia

\begin{alltt}
Class: Grupo \( \equiv \) foaf:Group
SubClassOf: Agent
Descrição: \emph{Classe que é pai de diferentes outras classes que definem grupos de pessoas.}
Relações:
  membros \( \equiv \) member

Class: Grupo_de_Pesquisa
SubClassOf: Grupo
Descrição: \emph{Classe dos grupos de pesquisa científica.}

Class: Organização \( \equiv \) foaf:Organization
SubClassOf: Agent
Descrição: \emph{Define uma organização, que pode ser uma Universidade, por
exemplo.}
Relações:
  faz_parte \( \equiv \) sub_organizacao: \emph{Uma \textbf{Organização}
pode fazer parte de outra \textbf{Organização}, por exemplo: o
Instituto de Física faz parte da USP.}
  possui \( \equiv \) super_organizacao: \emph{Uma \textbf{Organização} pode
possuir outra \textbf{Organização}, por exemplo: A USP possui o
Instituto de Matemática e Estatística.}

Class: Universidade
SubClassOf: Organização
Descrição: \emph{Classe das universidades e instituições
de ensino.}

Class:Instituto
SubClassOf: Organização
Descrição: \emph{Classe dos institutos de ensino e pesquisa e faculdades.}

Class: Departamento
SubClassOf: Organização
Descrição: \emph{Classe dos departamentos ligados a institutos de pesquisa ou
faculdades.}

Class: Revista
SubClassOf: Organização
Descrição: \emph{Classe das revistas científicas que publicam artigos.}
Relações:
  publicou: \emph{Uma \textbf{Revista} pode publicar um \textbf{Artigo}.
É o inverso da relação \textbf{publicado_em}.}

Class: Pessoa \( \equiv \) foaf:Person
SubClassOf: Agent
Descrição: \emph{Classe-pai dos tipos de pessoa.}
Relações:
  cursou: \emph{Uma \textbf{Pessoa} pode cursar algum \textbf{Curso}.}
  estudou_em: \emph{Uma \textbf{Pessoa} pode ter estudado em alguma
\textbf{Universidade, Instituto, Faculdade}.}
  membro_de: \emph{uma \textbf{Pessoa} pode ser membro de um  \textbf{Grupo}.
É o inverso da relação \textbf{membro}.}
  autor: \emph{Uma \textbf{Pessoa} pode ser autora de um  \textbf{Documento}.
É o inverso da relação \textbf{tem_autores}.}
  trabalhou_em: \emph{uma \textbf{Pessoa} pode trabalhar ou ter trabalhado
em uma \textbf{Organização}.}

Class: Aluno
SubClassOf: Pessoa
Descrição: \emph{Classe dos indivíduos que são alunos de algum curso.}
Relações:
  tem_orientador: \emph{Um \textbf{Aluno} pode ser orientado por um
\textbf{Professor} ou outro pesquisador.}

Class: Professor
SubClassOf: Pessoa
Descrição: \emph{Classe dos professores.}
Relações:
  orienta: \emph{Um \textbf{Professor} pode orientar um
\textbf{Aluno}. É o inverso da relação \textbf{tem_orientador}.}
  tem_orientador: \emph{Um \textbf{Professor} pode ser orientado por
um outro \textbf{Professor}, mesmo que no passado.}

Class: Curso
SubClassOf: Thing
Descrição: \emph{Classe dos cursos de graduação e pós-graduação.}
Relações:
  cursado_por: \emph{Um \textbf{Curso} pode ser cursado por \textbf{Pessoas}.}
  tipo_de_curso: \emph{Um \textbf{Curso} pode ser de Pós-Graduação ou
Graduação.}
  oferecido_por: \emph{Pode ser oferecido por uma instituição.}

Class: Graduação
SubClassOf: Tipo de Curso
Descrição: \emph{Classe dos cursos de graduação.}

Class: Pós Graduação
SubClassOf: Tipo de Curso
Descrição: \emph{Classe dos cursos de pós-graduação.}

Class: Documento \( \equiv \) foaf:Document
SubClassOf: Thing
Descrição: \emph{Classe dos documentos produzidos por alunos,
professores e pesquisadores.}
Relações:
  tem_autores: \emph{Um documento possui um ou mais autores do tipo
\textbf{Pessoa}. Essa propriedade é importante pois indica uma relação de \textbf{coautoria}.}

Class: Publicacao
SubClassOf: Documento
Descrição: \emph{Classe dos artigos científicos e outras publicações de
revistas, simpósios e outros eventos.}
Relações:
  publicado_em: \emph{Um \textbf{Publicacao} pode ser publicado em
uma \textbf{Revista} ou um \textbf{Evento}.}

Class: Tese
SubClassOf: Documento
Descrição: \emph{Classe-pai das Teses científicas de doutorado e dissertações de mestrado.}

Class: Eventos
SubClassOf: Thing
Descrição: \emph{Classe-pai dos eventos científicos, conferências, e simpósios.}
Relações:
  publicou: \emph{Um \textbf{Evento} pode ter publicações de \textbf{Artigos}.}

Class: foaf:Spatial Thing
Descricão: \emph{Classe dos indivíduos que possuem alguma
localização geográfica. \textbf{Evento} é uma delas.}

Classe: Local
SubClassOf: Thing
Descrição: \emph{Classe dos locais geográficos; são localizações.}
Relações:
  localizado \( \equiv \) foaf:based near \emph{Relaciona \textbf{Organização} ou \textbf{Evento} com um local
geográfico.}

Class: País
SubClassOf: Local

Class: Continente
SubClassOf: Local

\end{alltt}

A ontologia foi populada com as informações dos currículos lattes, e algumas consultas também farão parte da definição da ontologia. Essas consultas são escritas na linguagem SPARQL. Duas consultas que nos interessam agora respondem às seguintes perguntas:

\begin{itemize}
    \item Qual a área de pesquisa de um pesquisador?
    \item Quem são os pesquisadores que ele orientou?
    \item Há quantos anos um dado pesquisador trabalha em uma instituição?
\end{itemize}

A primeira questão de competência é extraída diretamente do currículo lattes. A segunda questão pode ser respondida através da relação \textit{orienta} das instâncias da classe \textit{Professor}. Um professor terá uma lista de orientandos, que também são pesquisadores.

Já a questão a respeito dos anos de trabalho em uma dada instituição

Conhecimento Prévio

O conhecimento prévio será extraído da ontologia através das consultas SPARQL, e transformado em uma matriz de relações, da seguinte forma: seja $P$ o conjunto de pesquisadores presentes na ontologia, e $R$ o conjunto de relações possíveis entre dois pesquisadores.  Cada linha dessa tabela resultante corresponde à um elemento do conjunto $P \times P$, e cada coluna indica as diversas possíveis relações entre um par de pesquisadores, assumindo o valor $1$ se a relação existir, e $0$ se não existir.

Como exemplo, seja $P = \{ p_1, p_2, p_3 \}$, o conjunto de relações $R = \{ orienta, coautor \}$, e sabemos que $p_1$ \texttt{orienta} $p_2$, pois $p_1$ foi orientador de $p_2$ em seu doutorado. E também sabemos que $p_1$ \texttt{coautor} $p_2$ e $p_1$ \texttt{coautor} $p_3$ e $p_2$ \texttt{coautor} $p_3$, pois os três publicaram um artigo juntos. A matriz resultante será:

\begin{table}[h!]
    \centering
    \begin{tabular}{|c|c|c|}
     \hline
      & orienta & coautor \\
     \hline\hline
     $p_1 p_2$ & 1 & 1  \\
     \hline
     $p_1 p_3$ & 0 & 1  \\
     \hline
     $p_2 p_3$ & 0 & 1  \\
     \hline
    \end{tabular}
    \caption{Matriz de relacionamentos entre pesquisadores}
    \label{matriz-relacoes}
\end{table}


%1. Estudar o domínio e modelar uma Ontologia.
%2. Modelar as consultas à ontologia que servirão para enriquecer o modelo de predição.
%3. Construir um modelo de predição de ligações baseado nos trabalhos da literatura, fazendo experimentos com diferentes técnicas de aprendizado supervisionado.
%4. Propor e implementar um algoritmo de extração de conhecimento da ontologia e enriquecimento do modelo de predição.
%5. Montar ambiente de testes e elaborar alguns testes de comparação.
