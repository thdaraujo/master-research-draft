%% ------------------------------------------------------------------------- %%
\chapter{Desenvolvimento}
\label{cap:desenvolvimento}

Neste capítulo, apresentamos o desenvolvimento do projeto e as principais práticas que o norteiam, através da discussão de um experimento exploratório em pequena escala (com uma quantidade pequena de dados) que servirá como estrutura inicial do projeto.

%% ------------------------------------------------------------------------- %%
\section{Estudo do domínio e construção de uma ontologia a respeito da Rede de Colaboração Científica}
\label{sec:desenvolvimento-ontologia}

A proposta consiste em rodar um experimento em pequena escala, de onde extraímos informações de um número pequeno de currículos Lattes, com interesse nas informações básicas dos pesquisadores do Departamento de Computação do Instituto de Matemática e Estatística da USP. Para isso, faremos uso dos seguintes dados: nome, área de atuação, publicações e coautores e local de trabalho ou residência.

Com base nessas informações, modelamos uma ontologia utilizando OWL através do software Protégé. A estrutura básica da ontologia serve para organizar o conhecimento a respeito desses pesquisadores e sua origem acadêmica, suas publicações e os relacionamentos que eles têm com outros. Com isso, podemos consultar essas informações estruturadas acerca da produção bibliográfica e científica dos docentes e alunos através de queries SPARQL. O conjunto de dados gerados por essas queries servirão de base para este experimento.

A ontologia Friend of a friend (FOAF), que descreve pessoas, relações e suas atividades, foi escolhida como ponto de partida devido o fato de ela ter alguns conceitos, classes e termos que nos auxiliam na modelagem da rede de colaboração científica.

As classes mais importantes da ontologia criada para este experimento são:

\begin{description}
  \item[Grupo] define grupos de pessoas (grupos de pesquisa, por exemplo).
  \item[Organização] define uma organização, que pode ser uma Universidade, por
  exemplo.
  \item[Universidade] \emph{(subclasse de \textbf{Organização})} classe das universidades e instituições de ensino e pesquisa.
  \item[Instituto] \emph{(subclasse de \textbf{Organização})} classe dos institutos de ensino e pesquisa e faculdades.
  \item[Departamento] \emph{(subclasse de \textbf{Organização})} classe dos departamentos ligados a institutos de pesquisa ou faculdades.
  \item[Pessoa] classe-pai dos tipos de pessoa.
  \item[Pesquisador] \emph{(subclasse de \textbf{Pessoa})} define pesquisadores, que são pessoas que publicam artigos e desempenham outras atividades científicas. Podem também ser alunos de cursos de pós-graduação, professores, e outros tipos de pessoas.
  \item[Publicação] define artigos científicos e outras publicações de
  revistas, simpósios e outros eventos.
  \item[Veículo] veículos de publicação de artigos científicos, como revistas, jornais, e eventos como simpósios e conferências.
  \item[Local] define os locais geográficos.
  \item[País] \emph{(subclasse de \textbf{Local})} classe dos países.
  \item[Cidade] \emph{(subclasse de \textbf{Local})} classe das cidades.
\end{description}

Algumas relações entre as entidades acima são importantes. Citamos algumas dessas relações abaixo:

\begin{description}
  \item[membro\_de] uma \textbf{Pessoa} pode ser membro de um \textbf{Grupo}. É o inverso da relação \textbf{membro}.
  \item[autor] uma \textbf{Pessoa} pode ser o autor de uma \textbf{Publicação}. É o inverso da relação \textbf{tem\_autores}.
  \item[trabalhou\_em] uma \textbf{Pessoa} pode trabalhar ou ter trabalhado em uma \textbf{Organização}.
  \item[publicou] um \textbf{Veículo} pode publicar uma \textbf{Publicação}, como um artigo. É o inverso da relação \textbf{publicado\_em}.
  \item[publicado\_em] uma \textbf{Publicação} pode ser publicada em
  um \textbf{Veículo} qualquer, como uma revista. É o inverso da relação \textbf{publicou}.
  \item[tem\_autores] uma \textbf{Publicação} possui um ou mais autores da classe
  \textbf{Pesquisador}. Essa propriedade é importante pois indica uma relação de
  \textit{coautoria} entre dois pesquisadores.
  \item[localizado] relaciona uma entidade como \textbf{Organização} ou \textbf{Veículo} com um local geográfico, como uma \textbf{Cidade}.
\end{description}


% TODO onto def
\begin{comment}

  \begin{alltt}
  \textit{Classe:} \textbf{Grupo} (subclasse de \textbf{Agent})
  \emph{Classe que define grupos de pessoas.}
  Relações:
  \begin{itemize}
  \item membros \( \equiv \) member
  \end{itemize}

  Class: Organização \( \equiv \) foaf:Organization
  SubClassOf: Agent
  Descrição: \emph{Define uma organização, que pode ser uma Universidade, por
  exemplo.}
  Relações:
    faz_parte \( \equiv \) sub_organizacao: \emph{Uma \textbf{Organização}
    pode fazer parte de outra \textbf{Organização}, por exemplo: o
    Instituto de Física faz parte da USP.}
    possui \( \equiv \) super_organizacao: \emph{Uma \textbf{Organização} pode
    possuir outra \textbf{Organização}, por exemplo: A USP possui o
    Instituto de Física.}

  Class: Universidade
  SubClassOf: Organização
  Descrição: \emph{Classe das universidades e instituições
  de ensino.}

  Class:Instituto
  SubClassOf: Organização
  Descrição: \emph{Classe dos institutos de ensino e pesquisa e faculdades.}

  Class: Departamento
  SubClassOf: Organização
  Descrição: \emph{Classe dos departamentos ligados a institutos de pesquisa ou
  faculdades.}

  Class: Revista
  SubClassOf: Organização
  Descrição: \emph{Classe das revistas científicas que publicam artigos.}
  Relações:
    publicou: \emph{Uma \textbf{Revista} pode publicar um \textbf{Artigo}. É o
    inverso da relação \textbf{publicado_em}.}

  Class: Pessoa \( \equiv \) foaf:Person
  SubClassOf: Agent
  Descrição: \emph{Classe-pai dos tipos de pessoa.}
  Relações:
    cursou: \emph{Uma \textbf{Pessoa} pode cursar algum \textbf{Curso}.}
    estudou_em: \emph{Uma \textbf{Pessoa} pode ter estudado em alguma
    \textbf{Universidade, Instituto, Faculdade}.}
    membro_de: \emph{uma \textbf{Pessoa} pode ser membro de um  \textbf{Grupo}.
    É o inverso da relação \textbf{membro}.}
    autor: \emph{Uma \textbf{Pessoa} pode ser autora de um  \textbf{Documento}.
    É o inverso da relação \textbf{tem_autores}.}
    trabalhou_em: \emph{uma \textbf{Pessoa} pode trabalhar ou ter trabalhado
    em uma \textbf{Organização}.}

  Class: Aluno
  SubClassOf: Pessoa
  Descrição: \emph{Classe dos indivíduos que são alunos de algum curso.}
  Relações:
    tem_orientador: \emph{Um \textbf{Aluno} pode ser orientado por um
    \textbf{Professor} ou outro pesquisador.}

  Class: Professor
  SubClassOf: Pessoa
  Descrição: \emph{Classe dos professores.}
  Relações:
    orienta: \emph{Um \textbf{Professor} pode orientar um
    \textbf{Aluno}. É o inverso da relação \textbf{tem_orientador}.}
    tem_orientador: \emph{Um \textbf{Professor} pode ser orientado por
    um outro \textbf{Professor}, mesmo que no passado.}

  Class: Curso
  SubClassOf: Thing
  Descrição: \emph{Classe dos cursos de graduação e pós-graduação.}
  Relações:
    cursado_por: \emph{Um \textbf{Curso} pode ser cursado por \textbf{Pessoas}.}
    tipo_de_curso: \emph{Um \textbf{Curso} pode ser de Pós-Graduação ou
    Graduação.}
    oferecido_por: \emph{Pode ser oferecido por uma instituição.}

  Class: Graduação
  SubClassOf: Tipo de Curso
  Descrição: \emph{Classe dos cursos de graduação.}

  Class: Pós Graduação
  SubClassOf: Tipo de Curso
  Descrição: \emph{Classe dos cursos de pós-graduação.}

  Class: Documento \( \equiv \) foaf:Document
  SubClassOf: Thing
  Descrição: \emph{Classe dos documentos produzidos por alunos,
  professores e pesquisadores, como um artigo científico.}
  Relações:
    tem_autores: \emph{Um documento possui um ou mais autores do tipo
    \textbf{Pessoa}. Essa propriedade é importante pois indica uma relação de
    \textbf{coautoria} entre dois pesquisadores.}

  Class: Publicacao
  SubClassOf: Documento
  Descrição: \emph{Classe dos artigos científicos e outras publicações de
  revistas, simpósios e outros eventos.}
  Relações:
    publicado_em: \emph{Uma \textbf{Publicação} pode ser publicada em
    uma \textbf{Revista} ou um \textbf{Evento}.}

  Class: Tese
  SubClassOf: Documento
  Descrição: \emph{Classe-pai das Teses científicas de doutorado e
  dissertações de mestrado.}

  Class: Eventos
  SubClassOf: Thing
  Descrição: \emph{Classe-pai dos eventos científicos, conferências, e simpósios.}
  Relações:
    publicou: \emph{Um \textbf{Evento} pode ter publicações de \textbf{Artigos}.}

  Class: foaf:Spatial Thing
  Descrição: \emph{Classe das entidades que possuem alguma
  localização geográfica. \textbf{Evento} é uma delas.}

  Classe: Local
  SubClassOf: Thing
  Descrição: \emph{Classe dos locais geográficos.}
  Relações:
    localizado \( \equiv \) foaf:based near \emph{Relaciona \textbf{Organização}
    ou \textbf{Evento} com um local geográfico.}

  Class: País
  SubClassOf: Local

  Class: Continente
  SubClassOf: Local

  Class: Cidade
  SubClassOf: Local

  \end{alltt}

\end{comment}

%% ------------------------------------------------------------------------- %%
\section{Consultas}
\label{sec:desenvolvimento-consultas}

A ontologia foi populada com as informações de currículos Lattes escolhidos, além de algumas consultas que também fizeram parte da definição da ontologia, sendo estas últimas feitas na linguagem SPARQL. As consultas que nos interessam agora respondem às seguintes questões:

\begin{itemize}
    \item Qual a área de pesquisa de um pesquisador?
    \item Quem são os pesquisadores que ele orientou?
    \item Há quantos anos esse pesquisador trabalha em uma instituição?
    \item Quem são as pessoas que colaboraram, como coautores, em sua produção científica?
\end{itemize}

A primeira questão de competência é extraída diretamente do currículo Lattes, quando possível. Trata-se da definição de suas áreas de atuação.

A segunda questão pode ser respondida através da relação \textit{orienta} das instâncias da classe \textit{Professor}. Um professor terá uma lista de orientandos, que, por sua vez, também são pesquisadores. A origem dessas informações é a seção de orientações do currículo Lattes. A terceira pode ser respondida através da análise da atuação profissional em instituições de ensino, pois ela contém o ano de início de atuação.

Por fim, a última questão é obtida fazendo-se uma consulta aos artigos publicados pelo pesquisador, aos trabalhos em eventos e aos demais tipos de produção técnica e outras informações bibliográficas que listam os nomes de outros colaboradores. O grande desafio nesse caso é fazer a desambiguação de nomes.

%% ------------------------------------------------------------------------- %%
\section{Predição de Ligações}
\label{sec:desenvolvimento-predicao}

Aplicaremos alguns modelos de predição, por exemplo, Árvores de Decisão e  Máquinas de Vetores de Suporte (SVM). O primeiro, por ser um modelo simples e ótimo para análise preliminar dos atributos relevantes que devem ser incluídos ou descartados, o que permite uma otimização do conjunto de atributos. Já os SVMs são importantes para nosso experimento, pois os trabalhos relacionados utilizaram esse modelo e obtiveram bons resultados. Outra vantagem é a possibilidade do uso de seleção de atributos (\textit{feature selection}), que também permite uma otimização e simplificação do conjunto de atributos que serão considerados pelo modelo.

Na fase A do experimento será construído um grafo com atributos similar ao descrito por \citet{Cervantes2014}, levando em consideração as ligações entre pesquisadores, o grau de cada vértice e outros atributos relacionados à estrutura da rede de colaboração. Esses atributos serão extraídos, gerando um conjunto de vetores de características, que chamaremos aqui de $VC$. Esse conjunto $VC$ será a entrada do modelo de predição na fase inicial do experimento, que também será enriquecido com conhecimento extraído da ontologia.

%% ------------------------------------------------------------------------- %%
\section{Extração do Conhecimento e Construção do \textit{background-knowledge}}
\label{sec:desenvolvimento-background-knowledge}

O conhecimento prévio, que chamaremos de $BK$, será extraído da ontologia através das consultas SPARQL e estruturado em uma matriz da seguinte forma: seja $P$ o conjunto de pesquisadores presentes na ontologia. Cada linha da tabela resultante corresponde a um elemento do conjunto de pares distintos de pesquisadores $Q = \{ (a, b) | a \in P \text{ e } b \in P \text{ e } a \neq b \}$, e cada coluna recebe o valor de algum atributo de $a$, ou de $b$, ou $R(a,b)$ (uma relação entre $a$ e $b$). $R(a,b) = 1$ se a relação existir ou for válida, e $R(a,b) = -1$ em caso contrário.

Como exemplo, admita um conjunto $P = \{ p_1, p_2, p_3 \}$ e algumas relações
como $R = \{ orienta, coautor \}$. Sabemos que $p_1$ \texttt{orienta} $p_2$, pois $p_1$ foi orientador de $p_2$ em seu doutorado. E também sabemos que $p_1$ \texttt{coautor} $p_2$ e $p_1$ \texttt{coautor} $p_3$ e $p_2$ \texttt{coautor} $p_3$, pois os três publicaram um artigo juntos. Também conhecemos alguns atributos desses pesquisadores: todos eles pertencem à área Ciência da Computação cujo identificador será, digamos, 123. Portanto, a matriz resultante será:

\begin{table}[h!]
    \centering
    \begin{tabular}{|c|c|c|c|}
     \hline
      & área & orienta & coautor   \\
     \hline\hline
     $Q(p_1, p_2)$ & 123 & 1  & 1  \\
     \hline
     $Q(p_1, p_3)$ & 123 & -1 & 1  \\
     \hline
     $Q(p_2, p_3)$ & 123 & -1 & 1  \\
     \hline
    \end{tabular}
    \caption{Matriz de \textit{background-knowledge} }
    \label{matriz-relacoes}
\end{table}

Entretanto, essa matriz pode ter um tamanho muito grande, dependendo da quantidade de pesquisadores. Uma solução possível seria limitar os exemplos, mostrando apenas os pares de pesquisadores que também possuem relação de coautoria. Entretanto, o impacto dessa alteração na eficácia do método ainda precisa ser investigado.

O conjunto de vetores de características $VC$, gerado na fase A, será copiado na fase B e enriquecido com as informações extraídas do conjunto $BK$. Chamamos aqui de enriquecimento a adição de novos atributos e relações aos elementos de $VC$. Cada pesquisador possui um conjunto de atributos relevantes, e esse conjunto será expandido com as informações do conjunto $BK$, gerando o conhecimento-prévio enriquecido, ou \textit{BK-enriched data} ($BK_e$).

O modelo de predição receberá como entrada esse conjunto de dados $BK_e$, separado aleatoriamente em subconjuntos de treinamento e teste. Após o treinamento, durante os testes, o modelo deverá prever se existe ou não uma relação de coautoria a partir dos atributos desse par de pesquisadores através da classificação de cada um dos exemplos. Esse resultado será posteriormente validado com o conjunto original, sendo possível com isso medir a eficiência do método.

%% ------------------------------------------------------------------------- %%
\section{Desenvolvimento e Testes}
\label{sec:desenvolvimento-testes}

A estrutura inicial do experimento segue o diagrama da \autoref{fig:diagrama-experimento}.

\begin{figure}[!h]
  \centering
  \includegraphics[width=.99\textwidth]{diagrama-menor}
  \caption{Diagrama de execução do experimento}
  \label{fig:diagrama-experimento}
\end{figure}

Ele está dividido em duas fases, a fase A e fase B. A fase A vai trabalhar apenas com dados relacionados à estrutura da rede, extraídos de um grafo. Já a B vai trabalhar com os dados sobre a estrutura da rede mais os dados extraídos da ontologia, que chamamos de conhecimento prévio. Os passos de cada fase serão discutidos a seguir.

Na fase A, um grupo pequeno de currículos Lattes no formato XML será lido e processado, e serão extraídos dados gerais desses currículos. A partir desses dados, será construído um grafo com atributos, onde cada nó representa um pesquisador, e cada aresta representa alguma relação entre os participantes dessa rede.

Desse grafo, será extraído um conjunto de vetores de características, que chamamos de $VC$, separado em conjuntos de treinamento e teste. O modelo de predição será treinado com o conjunto de treinamento, e será testado com o conjunto de teste. Essa fase vai gerar um relatório que lista a eficácia do modelo, pois o resultado da classificação pode ser validado com os dados iniciais.

Na fase B, o mesmo grupo de currículos será processado, utilizando esses dados para popular a ontologia descrita anteriormente. Serão feitas consultas a respeito das entidades e relações presentes na ontologia e o seu resultado será transformado em um conjunto de atributos, denominado \textit{background-knowledge}, ou $BK$, como descrito anteriormente.

Os vetores de características ($VC$) da fase A serão copiados e associados ao conjunto $BK$, processo que chamaremos de enriquecimento com conhecimento prévio, gerando o conjunto enriquecido $BK_e$. Dessa forma, outro modelo de predição será instanciado e receberá como estrada esses dados enriquecidos, também separados em conjunto de treinamento e teste. Será feito o treinamento do modelo, depois o teste deste modelo, seguido de uma análise do resultado.

Por fim, será feita a comparação da eficácia de ambos os modelos, verificando se há alguma melhora no método. Também será avaliado se os atributos adicionais extraídos de $BK_e$ são relevantes para o modelo, através de uma análise da importância relativa de cada atributo na classificação final (\textit{feature selection}).

Em suma: o ambiente de testes recebe um conjunto de currículos e um arquivo OWL correspondente à estrutura da ontologia, porém sem nenhuma instância. O software faz o processamento (fases A e B), e gera ao final um relatório comparando ambas as estratégias.


%% ------------------------------------------------------------------------- %%
\section{onto-mongo}
\label{sec:onto-mongo}

O \textit{onto-mongo}\footnotemark é um sistema desenvolvido para tratar o
problema de transformação de consultas SPARQL em consultas ao SGBD noSQL. Ele
transforma consultas SPARQL oriundas de alguma Ontologia OWL em consultas ao
MongoDB através de um mapeamento. O resultado desta consulta é transformada em
triplas RDF que são usadas pela ontologia.

\footnotetext{O projeto é de código aberto e o repositório está disponível no
github, em: \url{http://github.com/thdaraujo/onto-mongo} }

O protótipo possui as seguintes partes:
\begin{itemize}
    \item API
    \item Ontologia %TODO ?
    \item Mapeamento Ontologia x Coleções do Banco de Dados
    \item Tradutor de Consultas SPARQL em Consultas NoSQL
    \item Exportação
    \item Atualização da Ontologia %TODO ?
\end{itemize}

\subsection{API}
Uma API, ou Interface de Programação de Aplicações, é uma aplicação web que
possui um conjunto definido de requisições e respostas HTTP, e serve para
integrações entre sistemas ou serviços diversos. Em nosso caso, a API é um
software que recebe mensagens de entrada contendo uma consulta em linguagem SPARQL
e responde a essa consulta com um conjunto de triplas RDF extraídas de um banco
de dados.

Essa API foi desenvolvida em \textit{ruby}\footnotemark - uma linguagem
orientada a objetos - e utiliza o framework \textit{ruby on rails}\footnotemark,
criado para facilitar a criação de aplicações e serviços web.

O projeto foi desenvolvido utilizando docker, que é uma plataforma aberta com
várias ferramentas de virtualização para aplicações distribuídas, dotada dos
chamados \textit{containers}, que são pequenas máquinas virtuais linux que são
executadas de forma independente. Isto simplifica a configuração dos ambientes
de desenvolvimento e de aplicação, que são inicializados segundo um arquivo
de configuração, permitindo que sejam definidos os softwares que devem ser
instalados em cada máquina virtual e como devem estar configurados os ambientes.

Como exemplo, em nosso projeto foram criados dois \textit{containers}, um
para rodar a aplicação web (API) e o servidor, e outro para rodar o banco
de dados MongoDB.

A API possui classes escritas na linguagem \textit{ruby}. Essas classes
representam coleções presentes no banco de dados. Através destas classes,
é possível consultar o banco de dados, através do Mongoid.  A API possui
duas classes de modelo (\textit{model}) principais: \textit{Researcher},
e \textit{Publication}. A primeira diz respeito às informações do pesquisador,
a segunda se relaciona com publicações científicas.

\footnotetext{The Ruby Language \url{https://www.ruby-lang.org/} }
\footnotetext{Ruby on Rails \url{http://rubyonrails.org/} }

% TODO docker - onde colocar?

\subsection{Mapeamento}
A API possui classes escritas na linguagem \textit{ruby}, que é uma linguagem
orientada a objetos. Essas classes representam coleções presentes no banco de
dados onde são guardadas informações a respeito de pesquisadores e suas
publicações. Através destas classes, é possível consultar o banco de dados,
através do Mongoid.

Nosso mapeamento foi construído para ligar as classes da ontologia com as
classes do onto-mongo, para que fosse possível consultar informações
relacionadas a essas entidades no banco de dados de forma fácil. Utilizando essa
 abordagem, uma classe da ontologia é mapeada para uma classe ruby que
 representa ideias similares. As relações entre classes da ontologia também são
 mapeadas para relações entre classes da linguagem orientada a objetos.

O mapeamento é feito através da adição de alguns métodos de classe que indicam
qual relação entre classes e propriedades da ontologia equivalem a qual atributo
 na classe ruby, ou a qual relação entre classes da ontologia é equivalente a
 uma relação entre classes do onto-mongo.

Com isso, tornou-se possível mapear a estrutura presente na ontologia com a
estrutura das coleções presentes no banco de dados, passando pelas classes do
projeto, conforme mostrado a seguir:

% \lstinputlisting[language=Ruby, caption={Classe Researcher da API}, numbers=left]{files/researcher.rb}

% \lstinputlisting[language=Ruby, caption={Classe Publication da API}, numbers=left]{files/publication.rb}

Incluindo o módulo \textit{OntoMap} (linha 3 de ambas as classes) em uma classe
ruby como um \textit{mixin}, que é um tipo especial de herança múltipla, e que
provê novas funcionalidades à classe e o acesso aos métodos de mapeamento,
passando com isso a fazer parte da tradução. Já o método \textit{ontoclass}
(linha 9 de \textit{Researcher}) define qual classe da ontologia está
relacionada à classe \textit{Researcher}. No caso, ela se relaciona com a
classe \textit{foaf:Person} da ontologia.

O método \textit{maps} recebe dois parâmetros: \textit{from}, que diz respeito a
 algum relacionamento ou propriedade pertencente a uma classe da ontologia e
 \textit{to}, que indica qual atributo da classe \textit{Researcher} ou
 relacionamento com outra classe são equivalentes. Na linha 9, definimos um
 mapeamento entre a propriedade \textit{foaf:name} e o atributo \textit{name}.
 Na linha 10, a relação \textit{published} é mapeada para
 \textit{publications}, que representa o conjunto de publicações de um dado
 pesquisador. Importante ressaltar que uma instância da classe pesquisador
 pode possuir uma ou mais publicações.

O mapeamento aqui apresentado permitiu a tradução das consultas SPARQL em
consultas MongoDB sem a necessidade de haver um mapeamento direto entre a
estrutura da ontologia com as coleções de documentos do banco de dados. Em
outros trabalhos da literatura, quando falamos de OBDA para SGBDs relacionais,
esse tipo de mapeamento se faz diretamente entre consultas SQL e classes da
ontologia. Nossa abordagem possui assim uma camada conceitual que faz a ligação
entre ontologia e os dados descrita em uma linguagem orientada a objetos.

\subsection{Tradutor}
O tradutor transforma consultas SPARQL em consultas ao SGBD, utilizando para
isso o mapeamento. A consulta SPARQL é primeiro transformada em uma
\textit{SPARQL Syntax Expression (S-Expression)} através da biblioteca
\textit{sxp}\footnotemark, pois isto facilita a tradução. A consulta é então
transformada em uma estrutura de dados parecida com uma lista aninhada que
contem as triplas RDF, filtros, agrupamentos, e variáveis que devem ser
retornadas para esta consulta.

\footnotetext{Disponível em: \url{https://github.com/dryruby/sxp.rb} }

\subsection{Exportação}
Os dados extraídos do banco de dados são transformados em triplas RDF, que são
enviadas como resposta à consulta SPARQL inicial. Com isso, é possível popular
uma ontologia ou consultar e analisar esses dados de forma agnóstica, isto é,
sem que o cliente precise saber da estrutura do banco de dados e como estão
armazenadas as informações.

% TODO falar do retorno para a ontologia ou protegé etc.



% TODO colocar o docker na implementação

