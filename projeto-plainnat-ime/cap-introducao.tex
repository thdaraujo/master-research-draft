%% ------------------------------------------------------------------------- %%
\chapter{Introdução}
\label{cap:introducao}

Texto texto texto texto texto texto texto texto texto texto texto texto texto
texto texto texto texto texto texto texto texto texto texto texto texto texto
texto texto texto texto texto texto texto.



%% ------------------------------------------------------------------------- %%
\section{Motivação}
\label{sec:motivacao}

Texto texto texto texto texto texto texto texto texto texto texto texto texto
texto texto texto texto texto texto texto texto texto texto texto texto texto
texto texto texto texto texto texto texto.


%% ------------------------------------------------------------------------- %%
\section{Objetivos}
\label{sec:objetivos}

O objetivo do presente trabalho é demonstrar que um modelo de predição de ligações em uma rede de colaboração científica, quando enriquecido com conhecimento prévio (\textit{background knowledge}) extraído de uma ontologia, se torna mais eficaz em sua tarefa de predição.
Para isso, espera-se que os seguintes passos sejam cumpridos:

\begin{itemize}
    \item Modelar uma Ontologia para representar o conhecimento a respeito de uma rede de colaboração científica, contendo informações sobre pesquisadores, publicações e colaborações, e instituições.
    \item Construir consultas em SPARQL, que utilizam regras de inferência na Ontologia, e cujo resultado seja utilizado como conhecimento prévio para enriquecer o modelo de predição. Algumas dessas consultas são:
    \begin{itemize}
        \item Qual a área de pesquisa de um pesquisador?
        \item Quais são os pesquisadores que foram orientados por um dado pesquisador?
        \item Há quantos anos um dado pesquisador freqüenta uma instituição?
    \end{itemize}
    \item Construir um modelo de predição de ligações, baseado no trabalho de \citet{Cervantes2014}.
    \item Enriquecer o modelo de predição com as informações extraídas da ontologia.
    \item Comparar a eficácia dos dois modelos.
    \item Propor novas consultas que possam ser utilizadas como conhecimento prévio e fazer novas comparações.
\end{itemize}

%% ------------------------------------------------------------------------- %%
\section{Metodologia}
\label{sec:metodologia}

Texto texto texto texto texto texto texto texto texto texto texto texto texto
texto texto texto texto texto texto texto texto texto texto texto texto texto
texto texto texto texto texto texto.

%% ------------------------------------------------------------------------- %%
\section{Contribuições}
\label{sec:contribucoes}

As principais contribuições deste trabalho são as seguintes:

\begin{itemize}
  \item Item 1. Texto texto texto texto texto texto texto texto texto texto
  texto texto texto texto texto texto texto texto texto texto.

  \item Item 2. Texto texto texto texto texto texto texto texto texto texto
  texto texto texto texto texto texto texto texto texto texto.

\end{itemize}

%% ------------------------------------------------------------------------- %%
\section{Organização do Trabalho}
\label{sec:organizacao_trabalho}

No Capítulo~\ref{cap:conceitos}, apresentamos os conceitos ... Finalmente, no
Capítulo~\ref{cap:conclusoes} discutimos algumas conclusões obtidas neste
trabalho. Analisamos as vantagens e desvantagens do método proposto ...

As sequências testadas no trabalho estão disponíveis no Apêndice
