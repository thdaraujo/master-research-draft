%% ------------------------------------------------------------------------- %%
\chapter{Introdução}
\label{cap:introducao}

Texto texto texto texto texto texto texto texto texto texto texto texto texto
texto texto texto texto texto texto texto texto texto texto texto texto texto
texto texto texto texto texto texto texto.



%% ------------------------------------------------------------------------- %%
\section{Motivação}
\label{sec:motivacao}

Texto texto texto texto texto texto texto texto texto texto texto texto texto
texto texto texto texto texto texto texto texto texto texto texto texto texto
texto texto texto texto texto texto texto.


%% ------------------------------------------------------------------------- %%
\section{Objetivos}
\label{sec:objetivos}

O objetivo do presente trabalho é demonstrar que um modelo de predição de ligações em uma rede de colaboração científica, quando enriquecido com conhecimento prévio (\textit{background knowledge}) extraído de uma ontologia, se torna mais eficaz em sua tarefa de predição.
Para isso, espera-se cumprir os seguintes passos:

\begin{itemize}
    \item Modelar uma Ontologia para representar o conhecimento a respeito de uma rede de colaboração científica, contendo informações sobre pesquisadores, publicações e colaborações, e instituições.
    \item Construir consultas em SPARQL, que utilizam regras de inferência na Ontologia, que serão utilizadas como conhecimento prévio. Algumas dessas consultas são:
    \begin{itemize}
        \item Qual a área de pesquisa de um pesquisador?
        \item Quem são os pesquisadores que  ele orientou?
        \item Há quantos anos um dado pesquisador trabalha em uma instituição?
    \end{itemize}
    \item Construir um modelo de predição de ligações baseado no trabalho de \citet{Cervantes2014}.
    \item Enriquecer o modelo de predição com o conhecimento prévio extraído da ontologia, e comparar a eficácia dos dois modelos.
    \item Propor novas consultas à ontologia que possam ser utilizadas como conhecimento prévio aplicado ao modelo, e fazer novos testes.
\end{itemize}

%% ------------------------------------------------------------------------- %%
\section{Metodologia}
\label{sec:metodologia}

Os objetivos traçados serão alcançados através da execução dos seguintes ítens:

\begin{enumerate}
    \item \textbf{Estudo dos resultados de \citet{Cervantes2014}}: \\
      O objetivo desse ítem é entender o modelo de representação das redes de colaboração como grafos com atributos e o modelo de predição utilizando aprendizado supervisionado para que possa ser implementado e enriquecido.
    \item \textbf{Estudo sobre modelagem de Ontologias e Consultas via SPARQL}: \\
      O objetivo desse ítem é  ...
    \item \textbf{Propor uma forma de extração de conhecimento prévio da ontologia}: \\
      O objetivo desse ítem é propor um método de extração das informações obtidas via consulta da ontologia, construção de uma base de conhecimento prévio, e utilização desse conhecimento no enriquecimento do modelo, via adição de novos atributos aos vértices da rede de colaboraçào científica .
    \item \textbf{Executar a ideia proposta}: \\
      Desenvolver um algoritmo que recebe como entrada um conjunto de dados sobre pesquisadores, suas publicações e seus coautores, que seja capaz de popular a ontologia e extrair dela o conhecimento prévio, enriquecer o modelo de predição, e executá-lo. %TODO falar da saída?
    \item \textbf{Realizar teste da ideia com informações da Plataforma Lattes}: \\
      Preparar um ambiente de testes, que contenha um conjunto de dados extraídos da Plataforma Lattes através do \textit{scriptLattes} e a ontologia. Popular a ontologia, e extrair o conhecimento prévio. Rodar o modelo de predição com e sem o enriquecimento com conhecimento, e comparar os resultados. %TODO ajustar
\end{enumerate}


%% ------------------------------------------------------------------------- %%
\section{Contribuições}
\label{sec:contribucoes}

As principais contribuições deste trabalho são as seguintes:

\begin{itemize}
  \item Item 1. Texto texto texto texto texto texto texto texto texto texto
  texto texto texto texto texto texto texto texto texto texto.

  \item Item 2. Texto texto texto texto texto texto texto texto texto texto
  texto texto texto texto texto texto texto texto texto texto.

\end{itemize}

%% ------------------------------------------------------------------------- %%
\section{Organização do Trabalho}
\label{sec:organizacao_trabalho}

No Capítulo~\ref{cap:conceitos}, apresentamos os conceitos ... Finalmente, no
Capítulo~\ref{cap:conclusoes} discutimos algumas conclusões obtidas neste
trabalho. Analisamos as vantagens e desvantagens do método proposto ...

As sequências testadas no trabalho estão disponíveis no Apêndice
